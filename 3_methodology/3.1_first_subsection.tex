Here you can describe your research question(s) (sub-questions) in more detail and also infer some hypotheses from it, which will be tested in the analysis part. If your thesis is lacking a concrete research question which should be answered in the end somehow, then one fundamental part of thesis writing is missing! So please thing about research questions and the corresponding hypothesis which you will examine. 

Furthermore, you can lay out the different methodological approaches you need to answer the research questions. 

\subsubsection{Machine Learning Based Thesis}
Describe the overall approach and also the training and test data, where you get this from, and what are the labels. Describe also the overall task and how it will be tackled. Hereby, you need to describe which algorithms (including parametrization) you are using and the theoretical fundamentals of it. Also give some arguments \textbf{why} you are using different algorithms and models. 

\subsubsection{Questionnaire/ Lab Experiment}
If you writing a thesis where you conduct a lab experiment, or design an online questionnaire, you have to describe the construction of the experiment, if needed the design of the experimental platform or the online questionnaire. Also mention which kind of resources you are using and how the participants are acquired. If you have any special requirements for participants, these should also appear here. 

If you conduct experiments/surveys with human participants, please do the self-evaluation of the ethics commission of faculty IV and add a paragraph in the end of chapter 3, that the experiment/survey does not contain any ethical issues. You can also add the self-evaluation in the appendix of the thesis. Here is the link \url{https://ethikkommission.eecs.tu-berlin.de/en/} \href{https://ethikkommission.eecs.tu-berlin.de/en/}{Click Here}

As always the chapter needs to have a conclusion (last paragraph of the chapter) where you quickly sum up the main insights from the chapter and give some outlook about what comes next. This is crucial in order to help the reader to memorize what are the most important aspects of the chapter and to create a red line throughout the whole thesis. 